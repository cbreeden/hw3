\section{Exercise 8}\label{exercise-8}

Using the next reaction method, simulate and plot a single trajectory of
the model in Example 6.3 with \(\kappa_1 = 200\), \(\kappa_2 = 10\),
\(d_M = 25\), \(d_p = 1\), an initial condition of 1 gene, 10 mRNA, and
50 protein molecules, and a terminal time of \(T = 8\).

This model yields the four reaction vectors: \[
\xi_1 = (0,1,0),\ \xi_2 = (0,0,1),\ \xi_3 = (0,-1,0),\ \xi_4 = (0,0,-1).
\]

\section{Exercise 11}\label{exercise-11}

Let

\[
\xi_1 = (-1, 1, 0) \ \text{and}\ \xi_2 = (0, -1, 1),
\]

and let \(X^\theta_n\) be a discrete time Markov chain on
\(\mathbb{Z}^3_{\geq 0}\) with the following transition probabilities:

\begin{align}
p_{\vec x, \vec x + \xi_1} &= \frac{\theta xy}{\theta xy + y} \\
p_{\vec x, \vec x + \xi_2} &= 1 - p_1(x,y,z),
\end{align}

Assume that \(X^\theta_0 = (100, 5, 0)\) and \(\theta = 0.05\). Let
\(f(X^\theta) = \left(X^\theta_1\right)_{100}\); that is the first
component of the process after 100 steps. We estimate
\(\frac{d}{d\theta} E\left[f(X^\theta)\right]\) using a number of
different techniques.

\subsection{Finite Difference Method}\label{finite-difference-method}

In this problem, we employ a centered finite difference method using the
estimator

\[
\hat \mu^{\theta, h}_k
      = \frac{f(X^{\theta + h/2}_k) - f(X^{\theta - h/2}_k)}{h}.
\]

In an attempt to reduce the variance, we employ the common random
variables technique; in which we will use the same sequence of uniform
random variables when generating relizations \(X^{\theta + h/2}_k\) and
\(X^{\theta - h/2}_k\). We took \(n = 10,000\) samples and obtained the
following results for various \(h\).

\begin{longtable}[]{@{}cccc@{}}
\toprule
h & \(\hat \mu^{\theta, h}\) & \(\sigma^2\) & Confidence\tabularnewline
\midrule
\endhead
0.01 & -306.59 & 18,230.86 & 2.56\tabularnewline
0.005 & -304.76 & 36,782.56 & 3.76\tabularnewline
0.001 & -296.61 & 215,947.83 & 9.11\tabularnewline
0.0005 & -298.46 & 507,757.71 & 13.97\tabularnewline
\bottomrule
\end{longtable}

\subsection{Likelihood Ratio Method}\label{likelihood-ratio-method}

For likelihood ratio method, we took \(n = 100,000\) samples.

\begin{longtable}[]{@{}ccc@{}}
\toprule
\(\hat Y^{\theta}\) & \(\sigma^2\) & Confidence\tabularnewline
\midrule
\endhead
-331.29 & 564,0541.81 & 46.55\tabularnewline
\bottomrule
\end{longtable}

While emplying a control variate with the weight function obtained using
the Likelihood Ratio method, we had a dramatic reduction in variance.

\begin{longtable}[]{@{}ccc@{}}
\toprule
\(\hat Y^\theta\) & \(\sigma^2\) & Confidence\tabularnewline
\midrule
\endhead
-302.56 & 190,379.28 & 8.55\tabularnewline
\bottomrule
\end{longtable}
