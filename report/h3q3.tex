\section{Exercise 7}\label{exercise-7}

Using Gillespie's algoirthm, simulate and plot a single trajectory of

\[
  S_1 + S_2 \mathrel{\mathop{\leftrightharpoons}\limits_1^{0.1}} S_3,
\]

up to time \(T = 2\) under the assumption that \(S_1(0) = 15\),
\(S_2(0) = 20\), and \(S_3(0) = 0\). Find a 95\% confidence interval for
\(E[X_3(2)]\) using 1,000 independent simulations of the process.

\subsection{Setup}\label{setup}

With this reaction network we have the intensity functions:

\begin{align*}
  \lambda_1(x) = x_1x_2 & \quad \text{for $S_1 + S_2 \mathrel{\mathop{\rightharpoonup}\limits^1} S_3$}, \\
  \lambda_2(x) = 0.1x_3 & \quad \text{for $\phantom{S_1 +\ }\,S_3 \mathrel{\mathop{\rightharpoonup}\limits^{0.1}} S_1 + S_2$},
\end{align*}

and the reaction vectors

\[
  \xi_1 = (-1, -1, 1), \quad \xi_2 = (1, 1, -1).
\]

\section{Exercise 8}\label{exercise-8}

Using the next reaction method, simulate and plot a single trajectory of
the model in Example 6.3 with \(\kappa_1 = 200\), \(\kappa_2 = 10\),
\(d_M = 25\), \(d_p = 1\), an initial condition of 1 gene, 10 mRNA, and
50 protein molecules, and a terminal time of \(T = 8\).

This model yields the four reaction vectors: \[
\xi_1 = (0,1,0),\ \xi_2 = (0,0,1),\ \xi_3 = (0,-1,0),\ \xi_4 = (0,0,-1).
\]

\section{Exercise 11}\label{exercise-11}

Let

\[
\xi_1 = (-1, 1, 0) \ \text{and}\ \xi_2 = (0, -1, 1),
\]

and let \(X^\theta_n\) be a discrete time Markov chain on
\(\mathbb{Z}^3_{\geq 0}\) with the following transition probabilities:

\begin{align}
p_{\vec x, \vec x + \xi_1} &= \frac{\theta xy}{\theta xy + y} \\
p_{\vec x, \vec x + \xi_2} &= 1 - p_1(x,y,z),
\end{align}

Assume that \(X^\theta_0 = (100, 5, 0)\) and \(\theta = 0.05\). Let
\(f(X^\theta) = \left(X^\theta_1\right)_{100}\); that is the first
component of the process after 100 steps. We estimate
\(\frac{d}{d\theta} E\left[f(X^\theta)\right]\) using a number of
different techniques.

\subsection{Finite Difference Method (Common Random
Variables)}\label{finite-difference-method-common-random-variables}

In this problem, we employ a centered finite difference method using the
estimator

\[
\Delta^{\theta, h}_k
      = \frac{f(X^{\theta + h/2}_k) - f(X^{\theta - h/2}_k)}{h}.
\]

After taking a number of simulation, we will take the sample average

\[
\mu^{\theta, h} = \frac{1}{N-1} \sum_{k=0}^{N} \Delta^{\theta, h}_k.
\]

In an attempt to reduce the variance, we employ the common random
variables technique; in which we will use the same sequence of uniform
random variables when generating relizations \(X^{\theta + h/2}_k\) and
\(X^{\theta - h/2}_k\). We took \(n = 10,000\) samples and obtained the
following results for various \(h\).

\begin{longtable}[]{@{}cccc@{}}
\toprule
h & \(\mu^{\theta, h}\) & \(\sigma^2\) & Confidence\tabularnewline
\midrule
\endhead
0.01 & -306.59 & 18,230.86 & 2.56\tabularnewline
0.005 & -304.76 & 36,782.56 & 3.76\tabularnewline
0.001 & -296.61 & 215,947.83 & 9.11\tabularnewline
0.0005 & -298.46 & 507,757.71 & 13.97\tabularnewline
\bottomrule
\end{longtable}

\subsection{Finite Difference Method (Independent Random
Variables)}\label{finite-difference-method-independent-random-variables}

Out of curiosity, I also tried employing the algorithm without using a
common sequence of uniform random variables for constructing
relaizations of \(X^{\theta + h/2}_k\) and \(X^{\theta - h/2}_k\). The
results were astonishing; the variance increased by upto a factor of 10.
The following results were taking from \(n = 10,000\) simulations.

\begin{longtable}[]{@{}cccc@{}}
\toprule
h & \(\mu^{\theta, h}\) & \(\sigma^2\) & Confidence\tabularnewline
\midrule
\endhead
0.01 & -312.09 & 246,520 & 9.73\tabularnewline
0.005 & -314.66 & 991,336 & 19.51\tabularnewline
0.001 & -339.00 & 24,931,389 & 97.87\tabularnewline
0.0005 & -417.40 & 99,077,950 & 195.09\tabularnewline
\bottomrule
\end{longtable}

\subsection{Likelihood Ratio Method}\label{likelihood-ratio-method}

For likelihood ratio method, we took \(n = 100,000\) samples.

\begin{longtable}[]{@{}ccc@{}}
\toprule
\(\hat Y^{\theta}\) & \(\sigma^2\) & Confidence\tabularnewline
\midrule
\endhead
-331.29 & 564,0541.81 & 46.55\tabularnewline
\bottomrule
\end{longtable}

While emplying a control variate with the weight function obtained using
the Likelihood Ratio method, we had a dramatic reduction in variance.

\begin{longtable}[]{@{}ccc@{}}
\toprule
\(\hat Y^\theta\) & \(\sigma^2\) & Confidence\tabularnewline
\midrule
\endhead
-302.56 & 190,379.28 & 8.55\tabularnewline
\bottomrule
\end{longtable}
